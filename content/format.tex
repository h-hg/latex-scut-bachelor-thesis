\chapter{格式说明}

\section{纸张}

标准 A4 纸,上下左右页面边距为 25mm。

论文封面(底)、摘要和目录实行单面打印;
论文主体部分(引言、正文、结论、参考文献、附录)实行双面打印。

\section{页眉页脚}

\subsection{页眉}

页眉标注从论文主体部分(绪论、正文、结论)开始。
页眉分奇、偶页标注,其中偶数页的页眉为“华南理工大学学士学位论文”;
奇数页的页眉为章序及章标题。
页眉的上边距为 15mm,在版心上边线加一行 1.0 磅粗的实线,其上居中打印页眉;

\subsection{页脚}

论文页码从主体部分(绪论、正文、结论)开始,直至“参考文献、附录、致谢”结束,用五号阿拉伯数字编连续码,页码位于页脚居中。
摘要、目录、图表清单、主要符号表用五号罗马数字编连续码,页码位于页脚居中。
封面不编入页码。
页脚的下边距为 15mm。

\section{封面}

校徽:高度 2.75 cm。

类型:初号,黑体,居中。

标题:二号,黑体,加粗,居中。

信息:小三号,宋体,加粗,两端对齐。


\section{摘要}

\subsection{中文}

标题:小二号,黑体,居中,单倍行距,段前、段后各 0.5 行,两字中间空 2 字符。

正文:小四号,宋体,1.5 倍行距,段首行空两个汉字。

关键词标题:小四号,黑体,不缩进。

关键词内容:小四号,宋体。关键词之间用 ; 隔开,最后一个关键词不用分隔符。

\subsection{英文}

标题:小二号,Times New Roman 字体,居中,单倍行距,段前、段后各 0.5 行。

正文:小四号,Times New Roman 字体,1.5 倍行距,段首行空两个汉字,两端对齐。

关键词标题:小四号,Times New Roman 字体,不缩进,加粗。

关键词内容:小四号,Times New Roman 字体,关键词之间用 , 隔开,最后一个关键词不用分隔符。

\section{目录}

标题:小二号,黑体,居中,两字之间空 2 字符。

摘要、Abstract、目录、各章标题、结论、参考文献、致谢等:黑体,四号。

其余:宋体,小四号,行距 1.5 倍

\section{标题与正文}

各章标题:同中文摘要标题,章节序号与标题之间空一字符。

总结标题、参考文献标题、致谢标题:同中文摘要标题。

各节一级标题:黑体,小三号,居左,单倍行距,段前、段后各 0.5 行。

各节二级标题:黑体,四号,居左,单倍行距,段前、段后各 0.5 行。

各节二级标题:黑体,小四号,居左,单倍行距,段前、段后各 0.5 行。

正文:1.5 倍行距,
中文:宋体,小四号,每段首行空 2 个汉字;
字母和阿拉伯数字(含引用的数字):Times New Roman 字体,小四号。

\section{公式}

公式一般居中书写,序号按章编排,如本公式为第二章第一个公式,则序号为 (2-1)。

\section{表}

标题:位于表的上方,一般居中,宋体,五号。

序号:按章编排,如第四章第一个表,则序号为“表 4-1”,序号与文字描述之间空一格。

网格:表格不加左、右列线。

内容:表内数字空缺的格内加 “—” 字线。中文使用宋体,五号。

\section{图}

标题:位于图的下方,一般居中,宋体,五号。

序号:按章编排,如第四章第一个图,则序号为“图 4-1”,序号与文字描述之间空一格。

子图的需要用 a)、b)等表示。

\section{注释}

论文中有个别名称或情况需要解释时,可加注说明,注释可用页末注(将注文放在加注页稿纸的下端)或篇末注(将全部注文几种在文章末尾),而不可用行中注(即注文夹在正文中)。


\section{附录}

标题:同中文摘要标题。

正文:同摘要中文正文。

\section{参考文献}

标题:同中文摘要标题。

正文:同摘要中文正文。

\begin{description}
    \item[专著:] [序号] 作者名. 书名[M]. 出版地:出版单位,出版年:引文页码. 
    \item[期刊:] [序号] 作者名. 题名[J]. 刊名,年,卷号(期号):所引用的文献在期刊中的起止页码. 
    \item[报纸:] [序号] 作者名. 题名[N]. 报刊名,年-月-日(版次) .
    \item[论文集:] 析出文献主要责任者.析出文献题名[A].原文献主要责任者(可选).原文献题名[C].出版地:出版者,出版年.起止页码.
    \item[专利:] [序号]专利所有者.专利题名[P].专利国别:专利号,发布日期.
    \item[技术标准:] [序号]标准代号,标准名称[S].出版地:出版者,出版年.
    \item[报告:] [序号]作者.文献题名[R].报告地:报告会主办单位,年份.
    \item[电子文献:] [序号]作者.电子文献题名[文献类型/载体类型].文献网址或出处,发表或更新日期/引用日期(任选). 
    \item[学位论文:] [序号] 作者名. 题名[D]. 授予单位所在地:授予单位,授予年.
    \item[外文参考文献:] 按语言所在国学术界通行的格式
\end{description}

参考文献作者三名以内的全部列出,四名以上的列前三名,中文后加“等”,英文后加“et al”。